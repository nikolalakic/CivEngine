\documentclass[a4paper, 11pt]{article}
%\documentclass{letter}
\usepackage{cmsrb}
\usepackage[T2A,OT2]{fontenc}
\usepackage[utf8]{inputenc}
\usepackage[colorlinks]{hyperref}
\usepackage[left=2.5cm,top=0.5cm,right=1cm, bottom=1.25cm]{geometry} % Document margins
\usepackage{graphicx}
\usepackage{tabularx}
\usepackage{multirow}
\usepackage{ragged2e}
\usepackage{hhline}
\usepackage{array}
\usepackage{amsmath}
\hypersetup{
    urlcolor=blue
}

\renewcommand{\figurename}{Slika}
\renewcommand{\tablename}{Tabela}

\begin{document}

\section*{Ulazni parametri potpornog zida} 

\begin{figure}[h]
    \centering
    \includegraphics[width=\textwidth, height=15cm]{../Graphics/RetainingWall1_geometry.png}
    \caption{Geometrijiske i mehani\v{c}ke karakteristike potpornog zida}
    \label{geometrija_zida}
\end{figure}

\newpage

\begin{center}
Tabelarni prikaz ulaznih parametara potpornog zida:
\end{center}

\begin{table}[h]
\centering
\begin{minipage}{0.40\textwidth}
\centering
\begin{tabular}{|c|c|c|}
\hline
Parametar & Vrednost & Jedinica mere \\
\hline
$B$ & $B_final$ & $m$ \\
\hline
$h_{w1}$ & $hw1$ & $m$ \\
\hline
$h_{w2}$ & $hw2$ & $m$ \\
\hline
$t_{s}$ & $t_s$ & $m$ \\
\hline
$b_{t}$ & $b_t$ & $m$ \\
\hline
$t_{b}$ & $t_b$ & $m$ \\
\hline
$H$ & $h_u$ & $m$ \\
\hline
$D_{f}$ & $Df$ & $m$ \\
\hline
$b_{h}$ & $b_h$ & $m$ \\
\hline
$b$ & $b_o$ & $m$ \\
\hline
$\beta_{q}$ & $betta_q$ & $\circ$\\
\hline
$\alpha_{c}$ & $alpha_c$ & $\circ$\\
\hline
\end{tabular}
\caption{Geometrijiski parametri potpornog zida}
\label{tab:geometrijiski_parametri}
\end{minipage}
\hfill
\begin{minipage}{0.40\textwidth}
\centering
\begin{tabular}{|c|c|c|}
\hline
Parametar & Vrednost & Jedinica mere \\
\hline
$\gamma_{k,1}$ & $gamma_k_1$ & $kN/m^3$ \\
\hline
$\gamma_{k,2}$ & $gamma_k_2$ & $kN/m^3$ \\
\hline
$\gamma_{k,1,w}$ & $gamma_k1w$ & $kN/m^3$ \\
\hline
$\gamma^{'}$ & $gamma_prime$ & $kN/m^3$ \\
\hline
$\phi_{1}$ & $phi_k_1$ & $\circ$ \\
\hline
$\phi_{2}$ & $phi_k_2$ & $\circ$ \\
\hline
$\sigma_{Rd}$ & $sigma_rd$ & $kPa$ \\
\hline
$\gamma_{c}$ & 25 & $kN/m^3$ \\
\hline
\end{tabular}
\caption{Mehani\v{c}ki parametri tla u sklopu potpornog zida}
\label{tab:mehanicki_parametri}
\end{minipage}
\end{table}

%\begin{center}
%Tabelarni prikaz optere\'cenja i koeficijenata sigurnosti:
%\end{center}


\begin{table}[h]
\centering
\begin{tabular}{|c|c|c|}
\hline
Parametar & Vrednost & Jedinica mere \\
\hline
$q$ & $ q_u $ & $kN/m^2$ \\
\hline
$\gamma_{g}$ & $ gammag $ & - \\
\hline
$\gamma_{q}$ & $ gammaq $ & - \\
\hline
$\gamma_{g, fav}$ & $ gamma_gfav $ & - \\
\hline
$\gamma_{g, stab}$ & $ gamma_gstb $ & - \\
\hline
$\gamma_{g, dstb}$ & $ gamma_gdstb $ & - \\
\hline
$\gamma_{q, stb}$ & $ gamma_qstb $ & - \\
\hline
$\gamma_{q, dstb}$ & $ gamma_qdstb $ & - \\
\hline
$\gamma_{r,h}$ & $ gamma_rh $ & - \\
\hline
\end{tabular}
\caption{Optere\'cenje $q$ i koeficijenti sigurnosti}
\label{tab:koeficijenti_sigurnosti}
\end{table}

\newpage

\section*{Stati\v{c}ki prora\v{c}un potpornog zida}

\subsection*{Pretpostavka o Rankinovoj teoriji ravnih preseka:}

\noindent Uslov \v{s}iroke pete:

\begin{align*}
b_{h} &\geq (H - t_{b}) \cdot \tan\left(45 - \frac{\phi'_{1,d}}{2}\right)\\
\phi'_{1,d} &= \arctan\left(\frac{\tan(\phi_{1})}{\gamma'_{\phi}}\right) = phi_k_1^\circ \quad \text{gde je } \gamma'_{\phi} = 1\\
b_{h} &\geq (h_u - t_b) \cdot \tan\left(45 - \frac{phi_k_1}{2}\right) = bh_calculated \text{ } [m]
\end{align*}
Usvojena \v{s}irina pete:
\begin{align*}
b_{h} = b_h \text{ } [m]
\end{align*}
Ukupna \v{s}irina temeljne spojnice:
\begin{align*}
B = b_{t} + t_{s} + b_{h} = b_t + t_s + b_h = B_final \text{ } [m]
\end{align*}

\subsection*{Prora\v{c}un uticaja koji deluju na potporni zid:}

\begin{figure}[h]
    \centering
    \includegraphics[width=\textwidth, height=15cm]{../Graphics/RetainingWall1_gravity.png}
    \caption{Vertikalni uticaji na potporni zid}
    \label{uticaji_zida}
\end{figure}

\begin{figure}[h]
    \centering
    \includegraphics[width=\textwidth, height=15cm]{../Graphics/RetainingWall1_forces.png}
    \caption{Horizontalni (kosi) uticaji na potporni zid}
    \label{uticaji_zida}
\end{figure}

\newpage

Te\v{z}ina tla $G_{s1}$:

\begin{align*}
G_{s1} = b_{t} \cdot (D_{f} - t_{b}) \cdot \gamma_{k2} = b_t \cdot (Df - t_b) \cdot gamma_k_2 = gs1 \text{ } [kN/m]
\end{align*}

Te\v{z}ina tla $G_{s2}$:

\begin{align*}
G_{s2} = \tan (\alpha_{c}) \cdot \frac{h_{w2}^2}{2} \cdot \gamma_{k,1} = \tan (alpha_c) \cdot \frac{hw2^2}{2} \cdot gamma_k_1 = gs2 \text{ } [kN/m]
\end{align*}

%Te\v{z}ina tla $G_{s3}$ (za $h_{w1} \neq 0$):

%\begin{align*}
%G_{s3} &= (B - b_{t} - t_{s} - \tan{(\alpha_{c})} \cdot h_{w2}) \cdot h_{w2} \cdot \gamma_{k,1} \\
%       &= (B_final - b_t - t_s - \tan{alpha_c} \cdot hw2) \cdot hw2 \cdot gamma_k_1 = gs3 \text{ } [kN/m]
%\end{align*}

Te\v{z}ina tla $G_{s3}$ (za $h_{w1} = 0$):

\begin{align*}
G_{s3} &= (B - b_{t} - t_{s} - \tan{(\alpha_{c})} \cdot (h_{w2} - t_{b})) \cdot (h_{w2} - t_{b}) * \gamma_{k,1} \\
	   &= gs_three_eq_zero \text{ } [kN/m]
\end{align*}

Te\v{z}ina tla $G_{s4}$:

\begin{align*}
G_{s4} &= \tan{\alpha_{c}} \cdot \frac{(b_{h} + b - t_{s})^2}{2} \cdot \gamma_{k,1} \\
	   &= \tan{alpha_c} \cdot \frac{(b_h + b_o - t_s)^2}{2} \cdot gamma_k_1 = gs4 \text{ } [kN/m]
\end{align*}

Te\v{z}ina tla $G_{s5}$:

\begin{align*}
G_{s5} &= \tan{(\alpha_{c})} \cdot \frac{(h_{w1} - t_{b})^2}{2} \cdot \gamma_{k,1,w} \\
       &= \tan{(alpha_c)} \cdot \frac{(hw1 - t_b)^2}{2} \cdot gamma_k1w = gs5 \text{ } [kN/m]
\end{align*}

%Te\v{z}ina tla $G_{s6}$ (za $h_{w1} \neq 0$):

%\begin{align*}
%G_{s6} &= b_{h} \cdot (H - t_{b} - h_{w2}) \cdot \gamma_{k,1,w} \\
%       &= b_h \cdot (h_u - t_b - hw2) \cdot gamma_k1w = gs6 \text{ } [kN/m]
%\end{align*}

Te\v{z}ina tla $G_{s6}$ (za $h_{w1} = 0$):

\begin{align*}
G_{s6} &= b_{h} \cdot (H - t_{b} - (h_{w2}-t_{b}) \cdot \gamma_{k,1,w} \\
       &= b_h \cdot (h_u - t_b - (hw2-t_b)) \cdot gamma_k1w = gs_six_eq_zero \text{ } [kN/m]
\end{align*}

Te\v{z}ina betonskog dela potpornog zida $G_{c1}$:

\begin{align*}
G_{c1} = gc1 \text{ } [kN/m]
\end{align*}

Te\v{z}ina betonskog dela potpornog zida $G_{c2}$:

\begin{align*}
G_{c2} = gc2 \text{ } [kN/m]
\end{align*}

Te\v{z}ina betonskog dela potpornog zida $G_{c3}$:

\begin{align*}
G_{c3} = gc3 \text{ } [kN/m]
\end{align*}



\end{document}